\documentclass{article}
\usepackage{graphicx}
\usepackage{float}
\title{Torsional Axisymmetric Core Oscillations Visualiser} 
\author{Ogheneovo Mclarry Eduiyovwiri}
\date{August 5, 2020}
\begin{document}
\maketitle
\section{Abstract}
TACO-VIS provides a simple set of Python visualisation tools for 2D flow velocity data from fluid planetary interiors. It is mainly intended for animating torsional wave models for publication and presentation purposes. TACO-VIS is a lightweight module built only upon the common numpy/matplotlib Python packages and is free to be used and modified as the user requires.

\section{Introduction}

The dynamics of liquid planetary cores is fundamental to planetary-scale phenomena such as the generation of a magnetic field.

Understanding the relevant processes is a complex taxk because the fluid motion includes a wide range of conventive and oscillatary behaviour.

\section{Problem}
Earth as a rapidly rotating planent has such important type of dynamic behaviour known as Longitudunal Torsonal Oscillations of concentric cylinders aligned with the rotation axis, each of which spans the height of a spherical core.
Before now, these oscillations can only be visualized by the standard 2D contour plots but do not communicate the geometry of the waves within the spherical core.

\section{Dataset}

In this report we will be using this software to write python script that produces the 3D animation of the Cox et al. (2013) azimuthal component of velocity, contour plots data to communicate the geometry of the waves within it's spherical core.

\section{Result}

In addressing the problem stated above, Taco-vis python package which takes core flow velocity data for torsional waves helps visualized as a series of rotating concentric cylinders in a either a full 3D visualization or a 2D as a slice through the equatoral plane as seen in figure 2.

\begin{figure}[H]
	    \centering
	        \includegraphics[height=3in]{myplot1.png}
		    \includegraphics[height=3in]{myplot2.png}
		        \caption{Contour plot of the velocity of torsional waves in an equatorially symmetric spherical shell.}\label{fig:mesh1}			    
\end{figure}
\begin{figure}[H]
	    \centering
	        \includegraphics[width=0.6\textwidth]{myplot3.png}
		    \includegraphics[width=0.6\textwidth]{myplot4.png} 
		        \caption{A 3D and 2D visualisation of data from Cox et al. (Cox et al., 2013) approximated by 15 cylinders. The azimuthal velocity scale shown is non-dimensional and shared by both figures}
			     \label{fig:mesh1}
\end{figure}

    
\section{Benefits of TACO-VIS include:}
	\begin{itemize}
		\item .The tools require only the commonly available Python packages numpy and matplotlib and the well maintained ffmpeg framework.

		\item .Generating an animation can be done in just a simple few lines of code, with examples provided.

		\item .All plots and movies are of publication grade, with user choices for the titles, resolution and frame rate of saved images/movie files.
		\item .The matplotlib animations often draw fast enough to be suitable to be viewed live (dataset depending) without the need to encode to a movie file first.

	\end{itemize}

\end{document}
